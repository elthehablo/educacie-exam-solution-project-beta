\documentclass{article}
\usepackage[utf8]{inputenc}
\usepackage[framemethod=tikz]{mdframed}

\usepackage{amsmath,amssymb}
\usepackage{physics}

\title{Calculus A 2020-04-06 solutions}

\newcounter{exercise}
\setcounter{exercise}{0}
\newcounter{subexercise}
\setcounter{subexercise}{0}

\newenvironment{ex}[1][]{
	\refstepcounter{exercise}\setcounter{subexercise}{0}\par\medskip
	\noindent\textbf{Opdracht~\theexercise. }\rmfamily
}{\medskip}
\newenvironment{subex}[1][]{
	\refstepcounter{subexercise}\par\medskip
	\noindent\textbf{(\alph{subexercise}) }\rmfamily
}{\medskip}
\newenvironment{sol}[1][]{
	\par\medskip
	\begin{mdframed}
		\textbf{Antwoord: }
}{\end{mdframed}\medskip}
\newenvironment{proof}[1][]{
	\par\medskip
	\noindent\textit{Bewijs:}
}{\null\hfill\(\blacksquare\)\medskip}
\author{Educacie}
\date{\today}


\newcommand{\h}[1]{\left(#1\right)}
\begin{document}

\maketitle
\begin{ex}
	Definieer de funtie \(f:\left(-\infty,-2\right)\to\mathbb{R}\) door \(f(x) = \exp\left(\dfrac{2 - x}{2 + x}\right)\).
\end{ex}
\begin{subex}
	Bepaal de afgeleide van \(f\).
\end{subex}
\begin{sol}
    Uit de ketting regel en quotient regel volgt dat
    \begin{align*}
		\dv{f}{x}
		&= \dv{x}\h{\exp\h{\dfrac{2 - x}{2 + x}}} = \exp\h{\dfrac{2 - x}{2 + x}}\dv{x}\h{\dfrac{2 - x}{2 + x}}\\
		&= \exp\h{\dfrac{2 - x}{2 + x}}\dfrac{-\h{2 + x} - \h{2 - x}}{\h{2 + x}^2} = -\exp\h{\dfrac{2 - x}{2 + x}}\dfrac{4}{\h{2 + x}^2}
	\end{align*}
\end{sol}
\begin{subex}
	Bepaal het bereik van \(f\).
\end{subex}
\begin{sol}
	De maxima en minima worden bereikt op kritieke punten of the uiteinden van het domein. De kritieke punten zijn gegeven door \(\dv*{f}{x} = 0\), we merken op dat deze punten niet bestaan. Daarom beschouwen we de limieten van de functie.
	\begin{align*}
		\lim_{x\to-\infty}f\h{x} &= \lim_{x\to-\infty}\exp\h{\dfrac{2 - x}{2 + x}} = \lim_{x\to-\infty}\exp\h{\dfrac{\flatfrac{2}{x} - 1}{\flatfrac{2}{x} + 1}} = e^{-1}\\
		\lim_{x\to-2}{f\h{x}} &= 
	\end{align*}
\end{sol}
\end{document}
